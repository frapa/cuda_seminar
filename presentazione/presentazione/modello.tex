\documentclass{beamer}
\usepackage{graphicx} %per immagini
\usepackage[italian]{babel} %per scrivere in italiano
%\usepackage{tocbibind}%[nottoc,numbib]
\usepackage{amsmath,lipsum,hyperref,url,wrapfig,amsfonts,amssymb,latexsym,amsthm,eucal,eufrak}%matematica
%\usepackage[applemac]{inputenc} %accenti e apostrofi
\usepackage{subfig, caption, enumitem}
\usepackage[italian]{babel}
\usepackage[latin1]{inputenc}
\usepackage[T1]{fontenc}
\title{\bf Silenzio Cosmico}
\author{Filippo Vignanelli}
\date{21 ottobre 2015}
\institute[]{Universit� degli studi di Milano-Bicocca \\ Scuola di Scienze\\
Corso di Laurea in Fisica}
%\logo{\includegraphics[width=15mm]{logo_scuola}}
\usetheme{AnnArbor} 
\usecolortheme{beaver} 
\usefonttheme{structureitalicserif}
\setbeamertemplate{navigation symbols}{}
\usepackage{color}
\useinnertheme{rounded}
%\useoutertheme{sidebar}
%\setbeamercovered{dynamic}
\theoremstyle{definition}
\newtheorem{definizione}{Definizione}
\theoremstyle{plain}
\newtheorem{teorema}{Teorema}
\theoremstyle{remark} 
\newtheorem{oss}{Osservazione} 
\usepackage{enumerate}

\begin{document}
%\begin{frame}
   %\maketitle
%\end{frame}

\begin{frame}%{\phantom{title page}} 
%\phantom{a}
	\begin{figure}
  		 \includegraphics[height=2cm]{logo_scuola}
		 \centering
 	\end{figure}
\begin{center}
        \textit{\scriptsize{Universit\`a degli Studi Milano-Bicocca}}\\ 
	\tiny{Scuola di Scienze}\\
	\tiny{Corso di Laurea in Fisica} 
\end{center}

\begin{center}
	\Large{\bf Silenzio Cosmico:\\Valutazione della Dose Ambientale in un Sito Sotterraneo}\\
	\vspace{1cm}
	\begin{itemize}
		\item \footnotesize \footnotesize{Relatore: Prof.ssa M. Pavan} \hspace{2cm} \footnotesize{Candidato: Filippo Vignanelli}  
		\item \footnotesize{Correlatore: Prof.ssa S. Capelli} \hspace{1.8cm} \footnotesize{n. matricola: 761973}
	\end{itemize}	

 %\hspace{3.2cm} \footnotesize{Candidato:}\\
	 %\hspace{3cm}
	%\footnotesize{Filippo Vignanelli}\\ 
	%\vspace{1cm}
	%\footnotesize{21 ottobre 2015}
\end{center}
\end{frame}

%\begin{frame}
   %\frametitle{}
   %\tableofcontents
%\end{frame}

\begin{frame}
\section{Introduzione}
\frametitle{Introduzione}
\textbf{Silenzio Cosmico} � un esperimento che si sta svolgendo in questi anni presso i Laboratori Nazionali del Gran Sasso.\\
L'obiettivo � quello di studiare quanto la radiazione ambientale influenzi i meccanismi molecolari.
\end{frame}

\begin{frame}
	Andamento del danno biologico rispetto alla dose equivalente.
\begin{figure}
\centering
\includegraphics[width=1\linewidth, height=0.8\textheight]{modelli}
\label{fig:modelli}
\end{figure}


\end{frame}	
\begin{frame}
	All'interno dei Laboratori Nazionali del Gran Sasso la radiazione � composta principalmente da neutroni e da raggi $\gamma$.\vspace{0.5cm}\\
	L'obiettivo di questa tesi � di valutare, tramite simulazioni Monte Carlo, la dose assorbita dovuta dai raggi $\gamma$ nelle colture cellulari all'interno dei Laboratori Nazionali del Gran Sasso.
\end{frame}
\begin{frame}
	Lo spettro utilizzato per misurare il rate di dose assorbita � stato misurato con un rivelatore a diodo al germanio.\\
	La misura � stata realizzata con una soglia di 5 keV e un fondo scala di 3 MeV.
	\begin{figure}
		\centering
		\includegraphics[scale=0.35]{flussofotoniunder}
		\label{fig:flussofotoniunderlog}
	\end{figure}
	
\end{frame}	
 \begin{frame}
	\section{Arby}
 		\frametitle{Arby}
  	\textbf{Arby} � un simulatore Monte Carlo, basato su GEANT4, predisposto per applicazioni fisiche a basse energie.
  	\end{frame}
  	\begin{frame}
  		Il primo incubatore � all'interno di una schermatura in ferro antico spessa 5 cm.\\
  		Le dimensioni sono:
  		\begin{itemize}
  			\item larghezza: 89 cm 
  			\item altezza: 75.6 cm 
  			\item profondit�: 70.3 cm 
  		\end{itemize}
  		Le pareti dell'incubatore sono composte da 1 cm di acciaio e da 25 cm di polistirene.
  			\begin{figure}
  				\centering
  				\includegraphics[scale=0.25]{incubatoreferro}
  				\caption{Modello 3-D dell'incubatore con la schermatura in ferro.}
  				\label{fig:incubatoreferro}
  			\end{figure}
  		\end{frame}
	
\begin{frame}
	Il secondo incubatore non ha la schermatura in ferro.\\
	Le dimensioni esterne sono:
	 \begin{itemize}
	 		\item larghezza: 83 cm 
	 		\item altezza: 100 cm 
	 		\item profondit�: 68 cm 
	\end{itemize}
	Le pareti dell'incubatore sono composte da 1 cm di acciaio e 25 cm di polistirene
		\begin{figure}
			\centering
			\includegraphics[scale=0.3]{incubatorenoferro}
			\caption{Modello 3-D dell'incubatore senza la schermatura in ferro.}
			\label{fig:incubatorenoferro}
		\end{figure}
	\end{frame}	


\begin{frame}
	Arby permette di definire un oggetto con il ruolo di Detector, in questo modo viene registrata l'energia che le particelle $\gamma$ depositano all'interno del volume del materiale. \\
	Definendo la sfera d'acqua come Detector si pu� misurare la dose assorbita.\vspace{0.5cm}
	Oppure un oggetto pu� essere definito come Interceptor in questo modo viene registrata l'energia dei raggi $\gamma$ che arrivano sulla superficie del materiale.\\
\end{frame}	

\begin{frame}
\section{Analisi Dati}
\frametitle{Analisi Dati}
Per misurare la dose assorbita � stato scritto un algoritmo seguendo la seguente formula:
\[ D = \frac{\sum E_{i}n_{i}}{V} \]
\begin{itemize}
	\item $V$ � il volume della sfera d'acqua.
	\item $\sum E_{i}n_{i}$  somma  ogni intervallo di energia moltiplicato per il numero di conteggi dello stesso intervallo.
	\end{itemize}
	Le particelle $\gamma$, durante le simulazioni, vengono generate in modo isotropo su una superficie sferica che contiene il laboratorio.
\end{frame}
\begin{frame}
Utilizzando l'incubatore senza la schermatura in ferro sono state generate 3$\times 10^{7}$ particelle ottenendo il seguente spettro. I raggi $\gamma$ generati sono monocromatici con un energia di 500 keV.
\begin{figure}
\centering
\includegraphics[scale=0.2]{500kev10cm}
\label{fig:500kev10cm}
\end{figure}
La dose assorbita � stata normalizzata rispetto al numero di particelle generate $N_{p}$.

\[ D = \left(2.4 \pm 0.2 \right) \times 10^{-9}\  nGy/N_{p} \]
\\
L'errore percentuale � del 8.3$\% $.
\end{frame}

\begin{frame}
la stessa simulazione � stata fatta generando $10^{8}$ ottenendo il seguente spettro:
\begin{figure}
\centering
\includegraphics[scale=0.2]{500kev10cmdef}
\label{fig:500kev10cmdef}
\end{figure}
	Il valore della dose �:
	\[ D = \left(2.5 \pm 0.1 \right) \times 10^{-9}\  nGy/N_{p} \]
	\\
	L'errore percentuale � del 4$\% $.
\end{frame}	

\begin{frame}
La prima tabella rappresenta le simulazioni con l'incubatore senza il ferro mentre la seconda riassume la simulazione con l'incubatore schermato dal ferro.
\begin{figure}
\centering
\includegraphics[scale=0.3]{tab1}
\label{fig:tab1}
\end{figure}
\begin{figure}
\centering
\includegraphics[scale=0.3]{tab2}
\label{fig:tab2}
\end{figure}
Per alcuni valori di energia non ci sono stati eventi rivelati allora viene riportato il limite al $90 \% $ sulla dose assorbita.
\end{frame}	
\begin{frame}
	\begin{figure}
\centering
\includegraphics[scale=0.35]{confronto1}
\label{fig:confronto}
\end{figure}
\begin{itemize}
	\item \colorbox[named]{blue}{ }\ \ rappresenta l'incubatore senza la schermatura in ferro
	\item \colorbox[named]{red}{ }\ \ rappresenta l'incubatore con la schermatura in ferro
\end{itemize}
Raggi $\gamma$ con alte energie (3-5 MeV)  contribuiscono maggiormente al valore della dose assorbita.	
\end{frame}
\begin{frame}
Il seguente istogramma rappresenta i flussi $\gamma$ stimati all'interno dei Laboratori Nazionali del Gran Sasso in corrispondenza delle energie monocromatiche che ho analizzato.
\begin{figure}
\centering
\includegraphics[scale =0.25]{istodefinitivo}
\label{fig:grafico}
\end{figure}
Tra 0 e 5 keV non si ha una valutazione del flusso.
\end{frame}

\begin{frame}
Bisogna capire quale sia l'effetto delle schermature degli incubatori. 
Definendo l'interno degli incubatori come Interceptor si � ottenuto lo spettro energetico all'interno.\\
In questo modo a parit� di particelle generate si pu� confrontare l'effetto delle due schermature.
 	\begin{figure}
 		\centering
 		\includegraphics[scale=0.25]{entrambi}
 		\label{fig:noferrointerno}
 	\end{figure}
 	
\end{frame}
  \begin{frame}
  	\frametitle{Misura della Dose}
  	Per le successive simulazioni si sono calcolati dei rate di dose.\\
  	La dose � stata calcolata esattamente nello stesso modo delle simulazioni precedenti ma � stata divisa rispetto al tempo.\\
  	L'unit� temporale, ottenuta dallo spettro, � stata misurata nel seguente modo:
  	\[ t = \frac{N}{4\pi r^{2} \int\phi} \ \ \ \ \ \ \left[ t \right] = day \]
  	\begin{itemize}
  	\item $N$: � il numero di fotoni generati	
  	\item $\int\phi$: � la somma del flusso di fotoni
  	\item $r$: � il raggio della superficie sferica usata per generare i raggi $\gamma$
  	\end{itemize}	
  \end{frame}
  \begin{frame}
 Per l'incubatore senza la schermatura in ferro si � ottenuto il seguente spettro: 	
  \begin{figure}
\centering
\includegraphics[scale=0.2]{dose_fotoni_under_log}
\label{fig:dose_fotoni_under_log}
\end{figure}
 Il rate di dose �:
\[ \boxed{ \dot{D}_{sim\ nf} = 53 \pm 1 \  nGy/Day \  = \  2.19 \pm 0.04 \ nGy/h} \]
  \end{frame}
  \begin{frame}
  	 Per l'incubatore con la schermatura in ferro si � ottenuto il seguente spettro: 
  	\begin{figure}
\centering
\includegraphics[scale=0.2]{dose_fotoni_under_log_ferro10cm}
\label{fig:dose_fotoni_under_log_ferro10cm}
\end{figure}
 Il rate di dose �:
 \[ \boxed{\dot{D}_{sim\ f} =33 \pm 1 \  nGy / Day = 1.34 \pm 0.03  \  nGy / h }\]
  	\end{frame}
  	\begin{frame}
  		\section{Conclusioni}
  		\frametitle{Conclusioni e Osservazioni}
  		Entrambe le simulazioni sono state fatte generando 460 $\times 10^{6}$ particelle che equivale ad una misurazione di 0.02 giorni.\\
  		\begin{itemize}
  		\item  $\dot{D}_{sim\ nf} = 53 \pm 1 \  nGy/Day \  = \  2.19 \pm 0.04 \ nGy/h$
  		\item	$\dot{D}_{sim\ f} =33 \pm 1 \  nGy / Day = 1.34 \pm 0.03  \  nGy / h $
  		\end{itemize}	
  		Il rapporto tra i due valori � di: $ \dot{D}_{sim\ f} \approx  0.6 \times  \dot{D}_{sim\ nf} $.\\
  		\vspace{1cm}
  		I risultati che l'esperimento Silenzio Cosmico riporta sono:
  			\begin{itemize}
  				\item	$\dot{D}_{esp\ f} = 3 \ nGy / h$
  			\end{itemize}	
  	\end{frame}
  	\begin{frame}
  		A questo punto andrebbe rifatta la simulazione della dose assorbita con una sfera d'acqua pi� piccola che rappresenta meglio una coltura cellulare.\\
  		Se il valore della dose assorbita dovesse aumentare  significa che il volume della sfera d'acqua rispetto all'energia che depositano i raggi $\gamma$ secondo lo spettro presente nei LNGS � troppo grande.\\
  	\end{frame}	
  	\begin{frame}
  Tramite il coefficiente lineare di attenuazione si pu� determinare il rapporto $N(5cm)/N_{0} = e^{-\mu x}$ che rappresenta la probabilit�  che un fascio di fotoni incidente attraversa uno strato di 5 cm d'acqua senza aver avuto interazioni.\\
  La distanza 5 cm rappresenta il raggio della sfera d'acqua utilizzata per le simulazioni.
  \begin{figure}
\centering
\includegraphics[scale=0.32]{grafico1.png}
\label{fig:rapH2O}
\end{figure}
  	\end{frame}	
  	\begin{frame}
  		\frametitle{Bibliografia}
  		\begin{figure}
\centering
\includegraphics[scale=0.3]{biblio}
\label{fig:bilio}
\end{figure}

  	\end{frame}	
\end{document}