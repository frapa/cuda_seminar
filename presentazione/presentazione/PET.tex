\linespread{1.5}
%---------------------------------
\begin{frame}
\frametitle{Tomografia PET}
\begin{itemize}

	\item Tecnica diagnostica non invasiva
	\\ \setlength\parindent{24pt}Tracciamento radionuclidi
	\item \textbf{Decadimento $\mathrm{\beta +}$} ($^{18}F~ \to~^{18} O ~ ~e^+ ~~ \nu_e)$
	\item Annichilimento positronio $\to$ \textbf{fotoni quasi \textit{back to back}	} 511 keV
\end{itemize}
\end{frame}
%---------------------------------
\begin{frame}
\frametitle{}
\framesubtitle{PET}
\begin{minipage}{0.35\textwidth}
\includegraphics[scale=2]{immagini/PET.jpg}
\end{minipage}	\hfill
\begin{minipage}{0.5\textwidth}
\begin{itemize}
	\item Ricostruzione \textbf{LOF} (\textit{Line Of Flight}) fotoni tramite coincidenze
		\item Compton interni a sorgente peggiorano l'immagine.
	\\	Possibilità di scartali con \textbf{selezione energie} 511 keV
\end{itemize}
\end{minipage}
\begin{minipage}{\linewidth}
\begin{itemize}
	\item Rivelatori con risoluzione temporale elevata (esempio: MCP)
	\\ \setlength\parindent{24pt} Possibilità di ricostruire posizione di decadimento \\ \setlength				\parindent{24pt}(\textbf{TOF-PET})
	\item Rivelatori MCP permettono migliori prestazioni degli attuali cristalli scintillatori
\end{itemize}
\end{minipage}
\end{frame}
%---------------------------------

